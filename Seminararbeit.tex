\documentclass[12pt, a4paper, oneside]{scrartcl}

\usepackage[utf8]{inputenc}
\usepackage[T1]{fontenc}
\usepackage[ngerman]{babel}

\usepackage[T1]{fontenc}
\usepackage{caladea} % close enough für Cambria :clueless:

\usepackage{geometry}

\usepackage[ngerman]{babel}
\usepackage{csquotes}
\usepackage[
    backend=biber,
    style=verbose-note,
    sorting=nyt,
    giveninits=true,
    uniquename=init,
    citestyle=authoryear,
]{biblatex}

\addbibresource{Quellen.bib}
    
\geometry{
    a4paper,
    top=2.5cm,
    bottom=2.5cm,
    left=2.5cm,
    right=2.5cm
}

\usepackage{setspace}
\onehalfspacing

\usepackage{ragged2e}

\usepackage{indentfirst}

\usepackage{graphicx}
\usepackage{amsmath}
\usepackage{amsfonts}
\usepackage{amssymb}

\usepackage{float}

\usepackage{xcolor}
\usepackage{hyperref}
\hypersetup{
    colorlinks=true,
    linkcolor=black,
    citecolor=black,
    filecolor=black,
    urlcolor=blue,
    pdftitle={Seminararbeit: Social Engineering und Phishing},
    pdfauthor={Jens Fröhlich},
    pdfsubject={Der Staat als Hacker (SOT82533)}
}

\usepackage[footwidth=paper]{scrlayer-scrpage}

\setcounter{tocdepth}{3}
\setcounter{secnumdepth}{3}

\titlehead{
    \centering
    \includegraphics[width=4cm]{tum_logo.png} \\
    \vspace{1cm}
    Technische Universität München \\
    Fakultät für Informatik
    \vspace{0.5cm}
}

\subject{Seminararbeit \\ im Rahmen des Seminars \\ "Der Staat als Hacker"}

\title{Social Engineering und Phishing}

\author{Jens Fröhlich}

\date{12. Juni 2025}

%----------------------------------------------------------------------------------------

\begin{document}
\pagenumbering{gobble}

\begin{titlepage}
    \thispagestyle{empty}
    \maketitle
    \vspace{2cm}
    \begin{center}
    \end{center}
\end{titlepage}

%----------------------------------------------------------------------------------------

\clearpage
\pagestyle{empty}
\tableofcontents

%----------------------------------------------------------------------------------------

\justify

\newgeometry{
    top=2.5cm,
    bottom=2.5cm,
    left=2cm,
    right=5cm,
    marginparwidth=4.5cm,
    marginparsep=0.5cm
}

\pagestyle{scrheadings}
\clearpairofpagestyles

\ihead{}
\chead{}
\ohead{}

\ifoot{}
\cfoot[\pagemark]{\pagemark}
\ofoot{}

\pagenumbering{arabic}
\setcounter{page}{2}

\section{Einleitung}

\subsection{Social Engineering und Phishing}
\textit{"Hallo Mama/Papa, ich bin's. Ich habe mein Handy verloren und 
brauche dringend Geld. Kannst du mir bitte 100 Euro überweisen? 
Ich kann dir später alles erklären."} - Diese Art an Nachricht wird wohl kaum einem
neu vorkommen und ist ein gutes Beispiel für eine Thematik, die uns heutezutage immer häufiger begegnet:
\textbf{Social Engineering}.
\par
Dabei liegt der komplette Fokus auf dem Mensch als Schwachstelle, um durch gezielte psychologische
Manipulation an vertrauliche Informationen zu gelangen. Dabei wird oft das Vertrauen des Opfers ausgenutzt, 
um es dazu zu bringen, sensible Daten preiszugeben oder bestimmte Handlungen vorzunehmen.\footcite{BSISocialEngineering}
\par
\textbf{Phishing} im Speziellen ist eine besonders verbreitete Form des Social Engineerings. 
Hier handelt es sich konkret um den Versuch, über gefälschte E-Mails, Webseiten oder 
Nachrichten an persönliche Daten zu gelangen.\footcite{BSIPhishing}\\

\subsection{Relevanz des Themas}

\subsubsection{Fallzahlen}
Durch die hohe Erfolgsquote von Social Engineering bzw. Phishing stehen diese diese perfiden 
Angriffsmethoden nicht zu Unrecht so häufig im Rampenlicht. Knapp zwei von drei von sogenannten 
Databreaches sind auf das Element Mensch zurückzuführen, wovon mehr als 80\% von diesen Breaches
auf Phishing zurückzuführen sind. Zudem sind laut SlashNext die Zahl an Phishing-Angriffen ist
seit 2021 um 49\% gestiegen. Besonders hervorzuheben hierbei ist, dass der Anstieg mit der 
Veröffentlichung von ChatGPT das 4000-fach erreicht hat.\footcite{HoxHunt_Report}
\par
Zudem hat IBM analysiert, dass in etwa 15\% aller Datenschutzverletzungen auf Phishing, und
dadurch auch Social Engineering, zurückzuführen sind.\footcite{IBM_Report}

\subsubsection{Wirtschaftliche Schäden und Auswirkungen auf Unternehmen und Privatpersonen}
Auch meldet IBM im Jahr 2024 ein ungefährer Schaden von 4.88 Millionen US-Dollar pro Vorfall.\footcite{IBM_Phishing}
Vor allem Unternehmen werden oft betroffen, da man dort oft klare Hierarchien hat und deswegen
die Angreifer gezielter bestimmte Personen impersonifizieren können und somit die Mitarbeiter
schneller dazu bringen können, sensible Daten preiszugeben oder Schadsoftware herunterzuladen.
Außerdem kann, wenn ein Unternehmen stark betroffen ist, ein enormer Vertrauensverlust gegenüber
der Öffentlichkeit entstehen, was sich negativ auf die Verkaufszahlen auswirken kann.
\par
Ein sehr bekanntes Bespiel dafür ist der Target (US-amerikanische Einzelhändler) Hack von 2013,
welcher (wenn auch nur indirekt) ungefähr 162 Millionen US-Dollar Schaden verursacht hat. Dadurch,
dass Kreditkarteninformationen von Milionen Kunden gestohlen wurden, hat Target sowohl einen extremen
Vertrauensverlust erlitten, als auch starke finanzielle Einbußen hinnehmen müssen.\footcite{Target_Breach}\\

\section{Definition von Social Engineering}

% TODO
\subsection{Grundprinzipien und Psychologie des Social Engineerings}
Die Ausnutzung von menschlichen Eigenschaften wie Vertrauen, Angst, Autorität, Hilfsbereitschaft 
und viele mehr stehen im Vordergrund beim \textbf{Social Engineering}. Durch gezielte Manipulation
oder Erpressung werden unzählige Opfer (oftmals ohne ihr Wissen) dazu gebracht gewisse Geldsummen
für vermeindliche Dienstleistungen zu überweisen, sensible Daten anzugeben oder gezielte Schutzmechanismen
(z.B. 2-Faktor-Authentifizierung) zu umgehen.\footcite{BSISocialEngineering}\\
Hier geht es keineswegs um die Umgehung von technischen Schutzmaßnahmen, sondern um das gezielte
Täschen vom Mensch als Schwachstelle. Durch bewährtes Probiere von bestimmten Angriffstechniken
verfeinern Angreifer ihre Methoden immer weiter, dass Menschen schneller (und häufiger) beeinflusst
werden können, um so auf Systeme zuzugreifen, oder an sensible Daten zu gelangen.
Deswegen hilft auch nur eine Sache gegen diese Art von Angriffen: \textbf{Aufklärung}.\\

% TODO
\subsection{Gängige Methoden}
Neben den Subgenres von Phishing und Whaling ist \textbf{Baiting} sehr verbreitet.
Hier geht es darum, dem Opfer falsche Versprechen zu geben. Sei es entweder ein Gewinnspiel,
welches angeblich gewonnen wurde oder sogar in physischer Form, wie z.B. ein USB-Stick,
der gezielt an einen gut ersichtlichen Ort gelegt wurde und mit Malware präpariert ist.\footcite{CS_10Arten}
\par
Bei \textbf{Quid pro Quo} wird darauf geachtet, Opfern attraktive Dienstleistungen anzubieten, um im Gegenzug
an sensible Daten (oder Geld) zu gelangen. Diese Strategie wurde im Netz vor allem durch den YouTuber \textit{Jim Browning} bekannt.
Durch seine Videos, in denen er sogenannte Call-Center aufdeckt und diese sukzessive inflitriert und
am Ende an die lokalen Behörden übergibt, wird deutlich, wie einfach es ist, Menschen zu manipulieren.\footcite{JB_YouTube}
\par
\textbf{Honeytrapping} ist eine gängige Methode um sich mit Menschen, die auf diversen Dating-Plattformen nach
ihrer Traumfrau suchen, anzufreunden und dadurch einen finanziellen Vorteil zu erlangen. Vor allem 
der Aufschwung von Künstlicher Intelligenz und Date-Portalen hat es den Angreifern umso leicht gemacht
gefälschte Profile zu erstellen und falsche Versprechen zu geben.\footcite{CS_10Arten}


\section{Definition von Phishing}

% TODO
\subsection{Definition und Abgrenzung als Teilbereich des Social Engineerings}
\begin{itemize}
  \item PLATZHALTER
\end{itemize}


% TODO
\subsection{Die Verschiedenen Arten von Phishing und ihre Ziele}
\begin{itemize}
  \item Spear Phishing, Whaling, Clone Phishing, Voice Phishing, Smishing, Vishing, Pharming, E-Mail Phishing, ..
  \item Ziele (Datenklau (Finanzen maybe, aber auch Anschrift), Identitätsdiebstahl, Malware-Verbreitung, ..)
\end{itemize}

\section{Bekannte Phishing-Skandale und ihre Folgen}

% TODO, TBD
\subsection{DNC-Hack (2016)}
\begin{itemize}
  \item PLATZHALTER
\end{itemize}

% TODO, TBD
\subsection{Twitter Spear Phishing Attack (2020)}
\begin{itemize}
  \item PLATZHALTER
\end{itemize}

% TODO
\subsection{Weitere Skandale}
\begin{itemize}
  \item PLATZHALTER
\end{itemize}

\section{Bezug zum Ethischen Hacking}

% TODO
\subsection{Social Engineering und Phishing als Werkzeuge im Penetration Testing}
\begin{itemize}
  \item PLATZHALTER
\end{itemize}

% TODO
\subsection{Aufklärung und Sensibilisierung durch simulierte Angriffe}
\begin{itemize}
  \item PLATZHALTER
\end{itemize}

% TODO
\subsection{Rechtliche und ethische Rahmenbedingungen}
\begin{itemize}
  \item PLATZHALTER
\end{itemize}

\section{Fazit und Ausblick}

% TODO
\subsection{Zusammenfassung}
\begin{itemize}
  \item PLATZHALTER
\end{itemize}

% TODO
\subsection{Bedeutung von Präventionsmaßnahmen und Sensibilisierung}
\begin{itemize}
  \item PLATZHALTER
\end{itemize}

% TODO
\subsection{Zukünftige Entwicklungen und Herausforderungen}
\begin{itemize}
  \item PLATZHALTER
\end{itemize}

%----------------------------------------------------------------------------------------
\clearpage
\printbibliography[title={Literaturverzeichnis}]

\end{document}