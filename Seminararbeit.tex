\documentclass[12pt, a4paper, oneside]{scrartcl}

\usepackage[utf8]{inputenc}
\usepackage[T1]{fontenc}
\usepackage[ngerman]{babel}

\usepackage[T1]{fontenc}
\usepackage{caladea} % close enough für Cambria :clueless:

\usepackage{geometry}

\usepackage[ngerman]{babel}
\usepackage{csquotes}
\usepackage[
    backend=biber,
    style=verbose-note,
    sorting=nyt,
    giveninits=true,
    uniquename=init,
    citestyle=authoryear,
]{biblatex}

\addbibresource{Quellen.bib}
    
\geometry{
    a4paper,
    top=2.5cm,
    bottom=2.5cm,
    left=2.5cm,
    right=2.5cm
}

\usepackage{setspace}
\onehalfspacing

\usepackage{ragged2e}

\usepackage{indentfirst}

\usepackage{graphicx}
\usepackage{amsmath}
\usepackage{amsfonts}
\usepackage{amssymb}

\usepackage{float}

\usepackage{xcolor}
\usepackage{hyperref}
\hypersetup{
    colorlinks=true,
    linkcolor=black,
    citecolor=black,
    filecolor=black,
    urlcolor=blue,
    pdftitle={Seminararbeit: Social Engineering und Phishing},
    pdfauthor={Jens Fröhlich},
    pdfsubject={Der Staat als Hacker (SOT82533)}
}

\usepackage[footwidth=paper]{scrlayer-scrpage}

\setcounter{tocdepth}{3}
\setcounter{secnumdepth}{3}

\titlehead{
    \centering
    \includegraphics[width=4cm]{tum_logo.png} \\
    \vspace{1cm}
    Technische Universität München \\
    Fakultät für Informatik
    \vspace{0.5cm}
}

\subject{Seminararbeit \\ im Rahmen des Seminars \\ ``Der Staat als Hacker'' \\ (SOT82533)}

\title{Social Engineering und Phishing}

\author{Jens Fröhlich}

\date{12. Juni 2025}

%----------------------------------------------------------------------------------------

\begin{document}
\pagenumbering{gobble}

\begin{titlepage}
    \thispagestyle{empty}
    \maketitle
    \vspace{2cm}
    \begin{center}
    \end{center}
\end{titlepage}

%----------------------------------------------------------------------------------------

\clearpage
\pagestyle{empty}
\tableofcontents

%----------------------------------------------------------------------------------------

\justify

\newgeometry{
    top=2.5cm,
    bottom=2.5cm,
    left=2cm,
    right=5cm,
    marginparwidth=4.5cm,
    marginparsep=0.5cm
}

\pagestyle{scrheadings}
\clearpairofpagestyles

\ihead{}
\chead{}
\ohead{}

\ifoot{}
\cfoot[\pagemark]{\pagemark}
\ofoot{}

\pagenumbering{arabic}
\setcounter{page}{2}

\section{Einleitung}

\subsection{Social Engineering und Phishing}
\textit{``Hallo Mama/Papa, ich bin's. Ich habe mein Handy verloren und 
brauche dringend Geld. Kannst du mir bitte 100 Euro überweisen? 
Ich kann dir später alles erklären.''} - Diese Art an Nachricht wird wohl kaum einem
neu vorkommen und ist ein gutes Beispiel für eine Thematik, die uns heutezutage immer häufiger begegnet:
\textbf{Social Engineering}.
\par
Dabei liegt der komplette Fokus auf dem Mensch als Schwachstelle, um durch gezielte psychologische
Manipulation an vertrauliche Informationen zu gelangen. Dabei wird oft das Vertrauen des Opfers ausgenutzt, 
um es dazu zu bringen, sensible Daten preiszugeben oder bestimmte Handlungen vorzunehmen (wie zum Beispiel
Geld überweisen).\footcite{BSISocialEngineering}
\par
\textbf{Phishing} im Speziellen ist eine besonders verbreitete Form des Social Engineerings. 
Hier handelt es sich konkret um den Versuch sich als Personen auszugeben, die sie garnicht sind,
und somit über gefälschte E-Mails, Telefonanrufe, Webseiten oder Nachrichten um dadurch ihre Opfer zu 
manipulieren.\footcite{BSIPhishing}\\

\subsection{Relevanz des Themas}

\subsubsection{Fallzahlen}
Durch die hohe Erfolgsquote von Social Engineering bzw. Phishing stehen diese perfiden 
Angriffsmethoden nicht zu Unrecht so häufig im Rampenlicht. Knapp zwei von drei 
Databreaches sind auf das Element Mensch zurückzuführen, wovon mehr als 80\% von diesen Angriffen
als Phishing eingestuft werden können. Zudem ist laut SlashNext die Anzahl an Phishing-Angriffen
seit 2021 um ungefähr 49\% gestiegen. Besonders hervorzuheben sei hierbei, dass seit der Veröffentlichung
von ChatGPT die Anzahl an Phishing-Angriffen um mehr als 4000\% angestiegen ist.\footcite{HoxHunt_Report}
\par
Allerdings hat IBM analysiert, dass etwa 15\% aller Datenschutzverletzungen auf Phishing, und
dadurch auch Social Engineering, zurückzuführen sind.\footcite{IBM_Report}
Dieser Unterschied kann dadurch erklärt werden, dass Phishing in Studien oft enger oder breiter 
gefächert werden kann. So hat IBM höhstwahrscheinlich nur die Fälle aufgenommen, in denen 
Phishing eine zentrale Rolle gespielt hat, während HoxFunt auch Fälle mit einbezieht,
in denen Phishing bzw. Social Engineering Teil des Angriffs, aber nicht (nur) die Hauptursache war.

\subsubsection{Wirtschaftliche Schäden und Auswirkungen auf Unternehmen und Privatpersonen}
Laut einer Studie von IBM verursachen größere Phishing-Angriffe im Jahr 2024 durchschnittlich 
einen Schaden von etwa 4,88 Millionen US-Dollar pro Vorfall.\footcite{IBM_Phishing}
Gerade Unternehmen sind besonders gefährdet, da die klaren Hierarchien es Angreifern ermöglichen, 
gezielt bestimmte Personen zu imitieren. Dadurch lassen sich Mitarbeitende oft leichter dazu 
verleiten, sensible Informationen preiszugeben oder unbewusst mit gutem Willen Schadsoftware herunterzuladen.
Außerdem kann, wenn ein Unternehmen stark betroffen ist, ein enormer Vertrauensverlust gegenüber
der Öffentlichkeit entstehen, was sich negativ auf die Verkaufszahlen auswirken kann.
\par
Ein sehr bekanntes Bespiel dafür ist der Target (US-amerikanische Einzelhändler) Hack von 2013,
welcher (wenn auch nur indirekt) ungefähr 162 Millionen US-Dollar Schaden mit sich gebracht hat. Dadurch,
dass Kreditkarteninformationen von Millionen Kunden gestohlen wurden, hat Target sowohl einen extremen
Vertrauensverlust erlitten, als auch starke finanzielle Einbußen hinnehmen müssen.\footcite{Target_Breach}\\

\section{Definition von Social Engineering}

% TODO
\subsection{Grundprinzipien und Psychologie des Social Engineerings}
Die Ausnutzung von menschlichen Eigenschaften wie unter anderem Vertrauen, Angst, Autorität, Hilfsbereitschaft 
stehen im Vordergrund beim \textbf{Social Engineering}. Durch gezielte Manipulation
oder Erpressung werden unzählige Opfer (oftmals ohne ihr Wissen) dazu gebracht, Geldsummen
für vermeindliche Dienstleistungen zu überweisen, sensible Daten anzugeben oder gezielte Schutzmechanismen
(z.B. 2-Faktor-Authentifizierung) außer Kraft zu setzen.\footcite{BSISocialEngineering}\\
Hier geht es keineswegs um die Umgehung von technischen Schutzmaßnahmen, sondern um das gezielte
Täuschung und Manipulation des Menschen als Schwachstelle stehen im Mittelpunkt, sodass technische Barrieren 
oft irrelevant werden. Angreifer verfeinern ihre Methoden kontinuierlich durch erprobte Angriffstechniken. 
Dadurch gelingt es ihnen immer schneller und häufiger, Menschen zu beeinflussen.
Deswegen hilft auch nur eine Sache gegen diese Art von Angriffen: \textbf{Aufklärung}.\\

% TODO
\subsection{Gängige Methoden}
Neben den Subgenre von Phishing ist \textbf{Baiting} sehr verbreitet.
Hierbei geht es darum, dem Opfer falsche Versprechen zu geben. So kann zum Beispiel ein Gewinnspiel vorgetäuscht werden,
welches angeblich gewonnen wurde. Baiting gibt es aber auch in physischer Form. Häufig legt ein Angreifer einen USB-Stick
an einen gut ersichtlichen Ort und präperiert diesen im Vorhinein mit Malware. So muss das Opfer diesen 
Stick nur an seinen Computer stecken und schon hat der Angreifer Zugriff auf das System.\footcite{CS_10Arten}
\par
Bei \textbf{Quid pro Quo} wird darauf geachtet, Opfern attraktive Dienstleistungen anzubieten, um im Gegenzug
an Geld (oder in manchen Fällen auch sensible Daten) zu gelangen. Diese Strategie wurde im Netz vor allem durch den YouTuber \textit{Jim Browning} bekannt.
Durch seine Videos, in denen er sogenannte Call-Center aufdeckt und diese sukzessive inflitriert und
am Ende an die lokalen Behörden übergibt, wird deutlich, wie einfach es ist, Menschen zu manipulieren.\footcite{JB_YouTube}
\par
Oft verbunden mit Quod pro Quo ist die sogneannte \textbf{Scareware}, bei der Angreifer versuchen, ihre Opfer 
durch Angst dazu zu bringen, Schadsoftwäre verkleidet als Sicherheitssoftware herunterzuladen. Dies kann 
durch einfache Pop-up Fenster geschiehen, die behaupten, das System wäre infiziert, oder durch gezielte
Anrufe, bei denen sich die Angreifer als Mitarbeiter bekannter Firmen ausgeben und behaupten, das Konto
vom Nutzer wäre kompromittiert worden.\footcite{Keeper_Scareware}
\par
\textbf{Honeytrapping} ist eine gängige Methode um sich mit Menschen, die auf diversen Dating-Plattformen nach
ihrer Traumfrau suchen, anzufreunden und dadurch einen finanziellen Vorteil zu erlangen. Vor allem 
der Aufschwung von Künstlicher Intelligenz und Date-Portalen hat es den Angreifern umso leicht gemacht
gefälschte Profile zu erstellen und falsche Versprechen zu geben.\footcite{CS_10Arten}

\section{Definition von Phishing}

% TODO
\subsection{Definition und Abgrenzung als Teilbereich des Social Engineerings}
\begin{itemize}
  \item PLATZHALTER
\end{itemize}


% TODO
\subsection{Die Verschiedenen Arten von Phishing und ihre Ziele}
\begin{itemize}
  \item Spear Phishing, Whaling, Clone Phishing, Voice Phishing, Smishing, Vishing, Pharming, E-Mail Phishing, ..
  \item Ziele (Datenklau (Finanzen maybe, aber auch Anschrift), Identitätsdiebstahl, Malware-Verbreitung, ..)
\end{itemize}

\section{Bekannte Phishing-Skandale und ihre Folgen}

% TODO, TBD
\subsection{Democratic National Committee (DNC)-Hack}
Ein sehr bekanntes Beispiel für einen Phishing-Angriff ist der DNC-Hack von 2015 und 2016, 
bei dem die demokratische Partei der USA Opfer eines Spear-Phishing-Angriffs wurde. 
Dabei wurden E-Mails an 300 Angehörige der demokratischen Partei oder der Clinton-Kampagne
verschickt, die von russischen Hackern stammten. Die zwei Gruppen, ``Fancy Bear'' und ``Cozy Bear'',
haben insgesamt 70GB von den Servern der Clinton-Kampagne und 300GB von den DNC-Servern gestohlen.
Diese Daten, zusammen mit 20.000 E-Mails wurden auf WikiLeaks veröffentlicht.\footcite{IDStrong_DNC}
\par
Im Januar 2017 haben sowohl das FBI, als auch die CIA, NSA und mehrere andere US-Geheimdienste
untersucht, ob die russische Regierung mit diesem Angriff die Präsidentschaftswahl 2016 beeinflussen 
im Wohle von Donald Trump beeinflussen wollte.\footcite{NYT_DNC}
\par
Obwohl es also keine klaren Beweise gibt, kann man schon davon ausgehen, dass der Angriff eine 
große Rolle bei der Beeinflussung der Wahl gespielt hat. Man kann also sagen, dass dieser Angriff 
nicht nur ein Beispiel für Social Engineering und Phishing ist, sondern auch für die weitreichenden 
politischen und gesellschaftlichen Auswirkungen, die politisch motivierte Angriffe haben können.

% TODO, TBD
\subsection{Twitter Spear Phishing Attack (2020)}
\begin{itemize}
  \item PLATZHALTER
\end{itemize}

% TODO
\subsection{Weitere Skandale}
\begin{itemize}
  \item PLATZHALTER
\end{itemize}

\section{Bezug zum Ethischen Hacking}

% TODO
\subsection{Social Engineering und Phishing als Werkzeuge im Penetration Testing}
\begin{itemize}
  \item PLATZHALTER
\end{itemize}

% TODO
\subsection{Aufklärung und Sensibilisierung durch simulierte Angriffe}
\begin{itemize}
  \item PLATZHALTER
\end{itemize}

% TODO
\subsection{Rechtliche und ethische Rahmenbedingungen}
\begin{itemize}
  \item PLATZHALTER
\end{itemize}

\section{Fazit und Ausblick}

% TODO
\subsection{Zusammenfassung}
\begin{itemize}
  \item PLATZHALTER
\end{itemize}

% TODO
\subsection{Bedeutung von Präventionsmaßnahmen und Sensibilisierung}
\begin{itemize}
  \item PLATZHALTER
\end{itemize}

% TODO
\subsection{Zukünftige Entwicklungen und Herausforderungen}
\begin{itemize}
  \item PLATZHALTER
\end{itemize}

%----------------------------------------------------------------------------------------
\clearpage
\printbibliography[title={Literaturverzeichnis}]

\end{document}