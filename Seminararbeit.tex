\documentclass[12pt, a4paper, oneside]{scrartcl}

\usepackage[utf8]{inputenc}
\usepackage[T1]{fontenc}
\usepackage[ngerman]{babel}

\usepackage[T1]{fontenc}
\usepackage{caladea} % close enough für Cambria :clueless:

\usepackage{geometry}

\usepackage[ngerman]{babel}
\usepackage{csquotes}
\usepackage[
    backend=biber,
    style=verbose-note,
    sorting=nyt,
    giveninits=true,
    uniquename=init,
]{biblatex}

\addbibresource{Quellen.bib}
    
\geometry{
    a4paper,
    top=2.5cm,
    bottom=2.5cm,
    left=2.5cm,
    right=2.5cm
}

\usepackage{setspace}
\onehalfspacing

\usepackage{ragged2e}

\usepackage{indentfirst}

\usepackage{graphicx}
\usepackage{amsmath}
\usepackage{amsfonts}
\usepackage{amssymb}

\usepackage{xcolor}
\usepackage{hyperref}
\hypersetup{
    colorlinks=true,
    linkcolor=black,
    citecolor=black,
    filecolor=black,
    urlcolor=blue,
    pdftitle={Seminararbeit: Social Engineering und Phishing},
    pdfauthor={Jens Fröhlich},
    pdfsubject={Der Staat als Hacker (SOT82533)}
}

\usepackage[footwidth=paper]{scrlayer-scrpage}

\setcounter{tocdepth}{3}
\setcounter{secnumdepth}{3}

\titlehead{
    \centering
    % \includegraphics[width=4cm]{tum_logo.png} \\
    \vspace{1cm}
    Technische Universität München \\
    Fakultät für Informatik
    \vspace{0.5cm}
}

\subject{Seminararbeit \\ im Rahmen des Seminars \\ "Der Staat als Hacker"}

\title{Social Engineering und Phishing}

\author{Jens Fröhlich}

\date{12. Juni 2025}

%----------------------------------------------------------------------------------------

\begin{document}
\pagenumbering{gobble}

\begin{titlepage}
    \thispagestyle{empty}
    \maketitle
    \vspace{2cm}
    \begin{center}
    \end{center}
\end{titlepage}

%----------------------------------------------------------------------------------------

\clearpage
\pagestyle{empty}
\tableofcontents

%----------------------------------------------------------------------------------------

\justify

\newgeometry{
    top=2.5cm,
    bottom=2.5cm,
    left=2cm,
    right=5cm,
    marginparwidth=4.5cm,
    marginparsep=0.5cm
}

\pagestyle{scrheadings}
\clearpairofpagestyles

\ihead{}
\chead{}
\ohead{}

\ifoot{}
\cfoot[\pagemark]{\pagemark}
\ofoot{}

\pagenumbering{arabic}
\setcounter{page}{2}

\section{Einleitung}
Jens Fröhlich hat dazu ein wunderschönen Zitiertest\footcite{JensTest}.

\subsection{Social Engineering und Phishing}
\begin{itemize}
  \item Kurzes anschneiden beider Begriffe
  \item Abgrenzung der Begriffe
\end{itemize}

% TODO
\subsection{Relevanz des Themas}
\begin{itemize}
  \item Warum ist Social Engineering und Phishing wichtig?
\end{itemize}

% TODO
\subsubsection{Fallzahlen}
\begin{itemize}
  \item Statistiken und Fallzahlen zu Social Engineering und Phishing
  \item Ganz kleine Analyse (Anstieg / Rückgang)
  \item Vergleiche mit anderen Cyberangriffen (besonders wichtig oder nicht?)
\end{itemize}

% TODO
\subsubsection{Wirtschaftliche Schäden und Auswirkungen auf Unternehmen und Privatpersonen}
\begin{itemize}
  \item Wirtschaftliche Schäden
  \item Auswirkungen auf Unternehmen und Privatpersonen
  \item Beispiele
  \item Vertrauensverlust eventuell?
\end{itemize}


\clearpage
\section{Definition von Social Engineering}

% TODO
\subsection{Grundprinzipien und Psychologie des Social Engineerings}
\begin{itemize}
  \item Gute Definition
  \item Psychologische Grundlagen + Angriffspunkte (Vertrauen, Neugier, Angst, Autorität, Hilfsbereitschaft, ..)
  \item Abgrenzung zu anderen Angriffen (vllt. Malware)
\end{itemize}

% TODO
\subsection{Gängige Methoden}
\begin{itemize}
  \item Techniken (Baiting, Pretexting, Quid pro Quo, Tailgating, Vishing, Smishing, ..)
  \item Nicht Phishing vorwegnehmen
\end{itemize}

\clearpage
\section{Definition von Phishing}

% TODO
\subsection{Definition und Abgrenzung als Teilbereich des Social Engineerings}
\begin{itemize}
  \item PLATZHALTER
\end{itemize}

% TODO
\subsection{Die Verschiedenen Arten von Phishing und ihre Ziele}
\begin{itemize}
  \item Spear Phishing, Whaling, Clone Phishing, Voice Phishing, Smishing, Vishing, Pharming, E-Mail Phishing, ..
  \item Ziele (Datenklau (Finanzen maybe, aber auch Anschrift), Identitätsdiebstahl, Malware-Verbreitung, ..)
\end{itemize}

\clearpage
\section{Bekannte Phishing-Skandale und ihre Folgen}

% TODO, TBD
\subsection{DNC-Hack (2016)}
\begin{itemize}
  \item PLATZHALTER
\end{itemize}

% TODO, TBD
\subsection{Twitter Spear Phishing Attack (2020)}
\begin{itemize}
  \item PLATZHALTER
\end{itemize}

% TODO
\subsection{Weitere Skandale}
\begin{itemize}
  \item PLATZHALTER
\end{itemize}

\clearpage
\section{Bezug zum Ethischen Hacking}

% TODO
\subsection{Social Engineering und Phishing als Werkzeuge im Penetration Testing}
\begin{itemize}
  \item PLATZHALTER
\end{itemize}

% TODO
\subsection{Aufklärung und Sensibilisierung durch simulierte Angriffe}
\begin{itemize}
  \item PLATZHALTER
\end{itemize}

% TODO
\subsection{Rechtliche und ethische Rahmenbedingungen}
\begin{itemize}
  \item PLATZHALTER
\end{itemize}

\clearpage
\section{Fazit und Ausblick}

% TODO
\subsection{Zusammenfassung}
\begin{itemize}
  \item PLATZHALTER
\end{itemize}

% TODO
\subsection{Bedeutung von Präventionsmaßnahmen und Sensibilisierung}
\begin{itemize}
  \item PLATZHALTER
\end{itemize}

% TODO
\subsection{Zukünftige Entwicklungen und Herausforderungen}
\begin{itemize}
  \item PLATZHALTER
\end{itemize}

%----------------------------------------------------------------------------------------
\clearpage
\printbibliography[title={Literaturverzeichnis}]

\end{document}